%%%% Weekly Report Information %%%%
\newcommand{\handoutName}{Weekly report}
\newcommand{\handoutdate}{\today}
%\newcommand{\duedate}{}
% Header template used for Weekly Reports
\documentclass[11pt,twoside]{article}

\setlength{\oddsidemargin}{0pt}
\setlength{\evensidemargin}{0pt}
\setlength{\textwidth}{6.5in}
\setlength{\topmargin}{0in}
\setlength{\textheight}{8.5in}
\setlength{\voffset}{0in}

\providecommand{\titlesize}{small}


\usepackage{graphicx}
%\usepackage{subfigure}
\usepackage{palatino}
%\usepackage{cmbright}
\newcommand{\myMargin}{1.00in}
%\usepackage[pdftex]{hyperref}
\usepackage[small,bf]{caption}
\usepackage{amsmath}
\usepackage[usenames,dvipsnames]{color}
\usepackage{fancyhdr}
\pagestyle{fancy}
\usepackage{datetime}
\usepackage{fancyvrb}
\usepackage{color}
\usepackage[\titlesize, compact]{titlesec}
\usepackage{multicol}
\usepackage{enumitem}
\usepackage{pdfpages}
\usepackage{mdwlist}


\usepackage[ruled]{algorithm}
\usepackage{algpseudocode}

\usepackage{caption}
\usepackage{subcaption}



\newdateformat{dashdate}{\THEYEAR-\twodigit{\THEMONTH}-\twodigit{\THEDAY}}
\def\Tiny{\fontsize{3pt}{3pt}\selectfont}

\providecommand{\handoutName}{Handout title}
\providecommand{\handoutdate}{Handout date}
\providecommand{\duedate}{}

\lhead{Meeting with Prof. Becker\\
Fall, 2014}
\chead{}
\rhead{ Shiva Shahrokhi\\
\handoutdate }
\lfoot{}
\cfoot{\thepage}
\rfoot{\dashdate \Tiny \textcolor{Gray}{\today}}
\renewcommand{\headrulewidth}{0.4pt}
\renewcommand{\footrulewidth}{0.4pt}

\begin{document}

\vspace{0.60in}
\begin{center}
{\Large\textbf{\handoutName}}\\
\vspace{0.03in}
\textbf{\duedate}\\
\end{center}

\newcommand{\todo}[1]{
  \textcolor{Red}{
    \begin{tabular}{|c|}
      \hline
      \em \large \bfseries todo: \normalfont \normalsize #1 \\
      \hline
    \end{tabular}}
}


\section{My \emph{Objectives} this week}
\begin{itemize}
\item Make a hexagon block with laser cutter
\item Save the failed images for seeing all the robots, and play with them
\item Adapt the code in matlab for following the maze and pushing the block
\item Make the algorithm better: do not push the block back.
\item Writing the problem.
\end{itemize}


\section{My \emph{Accomplishments} this week}
\begin{itemize}
\item I made a hexagon with wood.
\item I saved the failed images which didn't see all the robots, I played with them, not yet have the perfect solution, when they run even if the light is on and has not changed, they see sometimes all the robots and sometimes it misses some. I am still working with the failed images.
\item I am working on adapting the matlab code for following the maze, it is not yet working perfectly.
\item I also has changed the visualization, so the new videos the desired goal position is more observable.
\end{itemize}





\subsection{How to gather all the robots together when they are spread out?}
Large populations of micro- and nanorobots are being produced in laboratories around the world, with diverse potential applications in drug delivery and construction. These activities require robots that behave intelligently. Limited computation and communication rules out autonomous operation or direct control over individual units; instead we must rely on global control signals broadcast to the entire robot population. It is not always practical to gather pose information on individual robots for feedback control; the robots might be difficult or impossible to sense individually due to their size and location. 
However, it is often possible to sense global properties of the group, such as mean position and variance. In our experiments, we showed that controlling mean is possible, and controlling variance with some constraints is possible. One of the constraints was that we had assumed that the swarm is clustered together and the swarm was trying to keep the variance small. Also, if during the experiment, a fraction of the swarm makes a new branch and put the swarm to two or more clusters by hitting the obstacles in their path, we could not gather them together again, and our algorithm for controlling variance was failed. So the next problem is to design an algorithm that will gather all the robots to an specific region with global input so that we can have the minimum variance. \\

For a \emph{S} shaped maze it is easy to implement. Alg. ~\ref{alg:S-Shaped} shows how to gather all the particles of the swarm when we have a \emph{S} shaped maze. With the same algorithm, we have a  solution for a mirrored \emph{S} shaped maze. It is not an optimal way for a \emph{T} shaped map also, but works for it. The problem is: Is there a general way to gather all the particles in \emph{any} map?

\begin{algorithm}
\caption{Gather Robots in One Corner of an S-shaped or inverse S-shape Maze}\label{alg:S-Shaped}
\begin{algorithmic}[1]

\Ensure   The maze is \emph{S}-shaped
\While{$\sigma_x^2 > \sigma_{min}^2 \vee \sigma_y^2 > \sigma_{min}^2$}
\State Go Maximal Left
\State Go Maximal Down
\State Go Maximal Right
\State Go Maximal Down
\EndWhile


\end{algorithmic}
\end{algorithm}



\section{My \emph{Plan} for next week}

\begin{itemize}
\item Having a video with perfect sensing the robots Mean control
\item Having a video for block pushing experience

\end{itemize}

\subsection{Meeting with Dr. Becker  }

\begin{itemize}
\item Discuss about the Qualifying Exam time.
\item Discuss about the next Dallas conference goals.
\end{itemize}

\end{document}
