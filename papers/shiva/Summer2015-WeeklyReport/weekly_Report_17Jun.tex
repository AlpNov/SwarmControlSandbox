%%%% Weekly Report Information %%%%
\newcommand{\handoutName}{Weekly report}
\newcommand{\handoutdate}{\today}
%\newcommand{\duedate}{}
% Header template used for Weekly Reports
\documentclass[11pt,twoside]{article}

\setlength{\oddsidemargin}{0pt}
\setlength{\evensidemargin}{0pt}
\setlength{\textwidth}{6.5in}
\setlength{\topmargin}{0in}
\setlength{\textheight}{8.5in}
\setlength{\voffset}{0in}

\providecommand{\titlesize}{small}


\usepackage{graphicx}
%\usepackage{subfigure}
\usepackage{palatino}
%\usepackage{cmbright}
\newcommand{\myMargin}{1.00in}
%\usepackage[pdftex]{hyperref}
\usepackage[small,bf]{caption}
\usepackage{amsmath}
\usepackage[usenames,dvipsnames]{color}
\usepackage{fancyhdr}
\pagestyle{fancy}
\usepackage{datetime}
\usepackage{fancyvrb}
\usepackage{color}
\usepackage[\titlesize, compact]{titlesec}
\usepackage{multicol}
\usepackage{enumitem}
\usepackage{pdfpages}
\usepackage{mdwlist}


\usepackage[ruled]{algorithm}
\usepackage{algpseudocode}

\usepackage{caption}
\usepackage{subcaption}



\newdateformat{dashdate}{\THEYEAR-\twodigit{\THEMONTH}-\twodigit{\THEDAY}}
\def\Tiny{\fontsize{3pt}{3pt}\selectfont}

\providecommand{\handoutName}{Handout title}
\providecommand{\handoutdate}{Handout date}
\providecommand{\duedate}{}

\lhead{Meeting with Prof. Becker\\
Fall, 2014}
\chead{}
\rhead{ Shiva Shahrokhi\\
\handoutdate }
\lfoot{}
\cfoot{\thepage}
\rfoot{\dashdate \Tiny \textcolor{Gray}{\today}}
\renewcommand{\headrulewidth}{0.4pt}
\renewcommand{\footrulewidth}{0.4pt}

\begin{document}

\vspace{0.60in}
\begin{center}
{\Large\textbf{\handoutName}}\\
\vspace{0.03in}
\textbf{\duedate}\\
\end{center}

\newcommand{\todo}[1]{
  \textcolor{Red}{
    \begin{tabular}{|c|}
      \hline
      \em \large \bfseries todo: \normalfont \normalsize #1 \\
      \hline
    \end{tabular}}
}


\section{My \emph{Objectives} this week}
\begin{itemize}
\item Loading windows on my Macbook
\item Running Kilobots with windows
\item Writing and simulating the algorithm of setting positions for two robots
\item Adding a folder for Journal version of our IROS papers
\end{itemize}


\section{Two Robots Positioning}\label{sec:algorithm}


\subsection{Controlling Covariance}

In this algorithm we want to control two position of two robots using friction. For ease of proof, we assume that we want to \emph{x} position of the robots, and \emph{y} of the robots are not the same and will remain the initial $\Delta y $ as we go through the algorithm.
\begin{algorithm}
\caption{Getting desired X-space}\label{alg:XControl}
\begin{algorithmic}[1]
\Require Knowledge of the starting and ending positions  of the two robots $s_1$ (topper robot) \& $s_2$ (lower robot)  \& $e_1$  \& $e_2$  and $L$ length of the bottom wall. Current position of the robot will be showed as $r$.
\Ensure $\Delta y(t) = \delta y(0)$ 
\State $e_{1x} - e_{2x} = \Delta e_x$
\State $s_{1x} - s_{2x} = \Delta s_x$
\If {$\Deltae < 0 $ }
\State Move almost to right wall
\Else Move almost to left wall
\EndIf
\State Move to bottom
\If $\Delta e - \Delta x > 0 $
\State Update current position of the robots: $r_1$ , $r_2$
\State Go right for $min(\|\Delta e - \Delta x \|, L- r_1)$
\Else Go left for $-min(\|\Delta e - \Delta x \|, r_1)$
\EndIf 
\State Move $\epsilon$ up
\State Update current position of the robots: $r_1$ , $r_2$
\If $\Delta r_x = \Delta e_x$ 
\State Go to $e_1$, $e_2$
\State Return
\Else Do the algorithm again with the inputs of $r_1$, $r_2$, $e_1$, $e_2$
\EndIf



\end{algorithmic}
\end{algorithm}


\section{My \emph{Accomplishments} this week}

\subsection{\emph{Kilobots}}

\begin{itemize}
\item I installed windows on my machine, and was able to see robots moving around and also communicating with each other.
\end{itemize}

\subsection{\emph{Algorithms}}

\begin{itemize}
\item I wrote the algorithm with latex which works with all the possible positions that the robots may see.
\end{itemize}





\section{My \emph{Plan} for next week}

\begin{itemize}
\item Complete the algorithm with Mathematica
\item Trying to go to the light by installing Kilobotics platform.(Now our robots don't understand their libraries because of the k-team platform.
\end{itemize}

\subsection{Meeting with Dr. Becker  }

\begin{itemize}
\item To see if my algorithm work and getting comments out of it.
\end{itemize}

\end{document}
