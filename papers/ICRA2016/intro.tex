\section{Introduction}\label{sec:Intro}



Micro- and nano-robots can be manufactured in large numbers. Large numbers of micro robots are required in order to deliver sufficient payloads, but the small size of these robots makes it difficult to perform onboard computation.  Instead, these robots are often controlled by a global, broadcast signal. 
In our previous work we focused on a block-pushing task, where a swarm of robots pushed a larger block through a 2D maze. One surprising result was that humans that only knew the swarm's mean and covariance completed the task faster that humans who knew the position of every robot~\cite{Becker2013b}. 
Inspired by that work, we proved that we can control the mean position of a swarm and that with an obstacle we can control the swarm's position variance ($\sigma_x$ and $\sigma_y$). 
We then wrote automatic controllers which could complete a block pushing task, but these controllers had some limitations~\cite{ShahrokhiIROS2015}. 
One of the limitations was that we could only compress our swarm along the world $x$ and $y$ axes, and could not navigate workspaces with narrow corridors with other orientations. 
One solution to these problems would be a controller that regulates the swarm's position covariance, $\sigma_{xy}$. 
For controlling $\sigma_{xy}$, we prove that the swarm position covariance $\sigma_{xy}$ is controllable given boundaries with non-zero friction. 
We then prove that two orthogonal boundaries with high friction are sufficient to arbitrarily position a swarm of robots. 
We conclude by designing controllers that efficiently regulate $\sigma_{xy}$.

\begin{figure}
\centering
\begin{overpic}[width=0.9\columnwidth]{leaf.png}\end{overpic}
\caption{\label{fig:vascularNetwork}Vascular networks are common in biology such as the circulatory system and cerebrospinal spaces, as well as in porous media including sponges and pumice stone.  Navigating a swarm using global inputs, where each member receives the same control inputs, is challenging
due to the many obstacles. This paper demonstrates how friction with walls can be used to change the shape of a swarm.} %TODO: save this as a pdf
\end{figure}






% Our paper is organized as follows.  After a discussion of related work in Section \ref{sec:RelatedWork}, we describe our experimental methods for an online human-user experiment in Section \ref{sec:expMethods}.  We report the results of our experiments in Section \ref{sec:expResults}, discuss the lessons learned in Section \ref{sec:discussion}, and end with concluding remarks in Section \ref{sec:conclusion}.


