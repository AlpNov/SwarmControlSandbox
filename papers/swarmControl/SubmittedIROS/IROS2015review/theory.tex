%%%%%%%%%%%%%%%%%%%%%%%%%%%%%%%%%%%%%%%%%%%%%%%%%%%%%%%%%%%
\section{Theory}
\label{sec:theory}
%%%%%%%%%%%%%%%%%%%%%%%%%%%%%%%%%%%%%%%%%%%%%%%%%%%%%%%%%%%


% Theory:
%% Models (of each robot, of the global input)
%% Impossibility result
%% Controlling Mean proof
%% Controlling Variance proof (CLF)
%% A Hybrid controller using hysteresis
\subsection{Models}
Consider holonomic robots that move in the 2D plane. We want to control position and velocity of the robots. 
First, assume a noiseless system with just one robot.
 Our inputs are global forces $[u_x,u_y]$. \begin{equation}
\begin{bmatrix}
\dot{p}_x &=& v_x \\
\dot{v}_x &=& \frac{1}{m} u_x
\end{bmatrix},
\begin{bmatrix}
\dot{p}_y &=& v_y \\
\dot{v}_y &=&\frac{1}{m} u_y
\end{bmatrix}
\end{equation}

The state-space representation in standard form is: 
\begin{align}\label{eq:stdform}
\dot{\mathbf{x}}(t)  &=  A \mathbf{x}(t) + B \mathbf{u}(t) \\
\mathbf{y}(t) &= C \mathbf{x}(t) + D \mathbf{u}(t)\nonumber 
\end{align}


%where $x(t)$ represents our states, \emph{u(t)} is our input. %and \emph{e(t)} represents noise in the system.  We also have $y(t)$ as our output.\\
We define our state vector $\mathbf{x}(t)$ as:
\begin{align}
\left[ x_1,x_2,x_3,x_4\right]^\intercal = \left[ p_x,v_x,p_y,v_y\right]^\intercal, \nonumber
\end{align}

and our state space representation as:
\begin{equation}
\begin{bmatrix}
\dot{x}_1\\ 
\dot{x}_2\\
\dot{x}_3\\
\dot{x}_4
\end{bmatrix} = \begin{bmatrix}
0 & 1 & 0 & 0 \\
0 & 0 & 0 & 0\\
0 & 0 & 0 & 1\\
0 & 0 & 0 & 0
\end{bmatrix}  \begin{bmatrix}
x_1\\
x_2\\
x_3\\
x_4
\end{bmatrix} + \begin{bmatrix}
0 & 0 \\
\frac{1}{m} & 0 \\
0 & 0 \\
0 & \frac{1}{m}
\end{bmatrix} u
\end{equation}

We want to find number of states that we can control, which is given by the rank of the \emph{controllability matrix}
\begin{equation}
\mathcal{C} = \{ B, AB, A^2B, ... , A^{n-1}B \}.
\end{equation}

\begin{equation}
\textrm{Here }
\mathcal{C}=\left\{
\begin{bmatrix} 
0 & 0\\
\frac{1}{m} & 0 \\
0 & 0 \\
0 & \frac{1}{m}
\end{bmatrix}
,
\begin{bmatrix} 
\frac{1}{m}& 0\\
0 & 0\\
0 & \frac{1}{m}\\
0 & 0
\end{bmatrix}
 \right\}
\end{equation}
And thus all four states are controllable

\subsection{Independent control with multiple robots is impossible}
A single robot is fully controllable, but what happens with $n$ robots? For holonomic robots, movement in the $x$ and $y$ coordinates are independent, so for notational convenience without loss of generality we will focus only on movement in the $x$ axis. Given $n$ robots to be  controlled in the $x$ axis, there are $2n$ states, $n$ positions and $n$ velocities:
\begin{equation}\left[ x_1,x_2,\ldots, x_{2n-1},x_{2n}\right]^\intercal = \left[ p_{x,1},v_{x,1},\ldots,p_{x,n},v_{x,n}\right]^\intercal. \nonumber\end{equation}
%\begin{align}
%\dot{p}_{x1} &= v_{x1}\\\nonumber
%\dot{v}_{x1} &= a_{x1}\\\nonumber
%\dot{p}_{x2} &= v_{x2}\\\nonumber
%\dot{v}_{x2} &= a_{x2}\\\nonumber
%&\vdots\\\nonumber
%\dot{p}_{xn} &= v_{xn}\\\nonumber
%\dot{v}_{xn} &= a_{xn}\nonumber
%\end{align}
Our state-space representation is:
\begin{equation}
\begin{bmatrix}
\dot{x}_1\\ 
\dot{x}_2\\
\vdots\\
\dot{x}_{2n-1}\\
\dot{x}_{2n}

\end{bmatrix} = \begin{bmatrix}
0 & 1 & \ldots & 0 & 0 \\
0 & 0 & \ldots& 0 & 0 \\
\vdots &  \vdots & \ddots & \vdots & \vdots \\
0 & 0  & \ldots & 0 & 1 \\
0 & 0 & \ldots& 0 & 0 
\end{bmatrix}  \begin{bmatrix}
x_1\\
x_2\\
\vdots \\
x_{2n-1}\\
x_{2n}
\end{bmatrix} + \begin{bmatrix}
0\\
1\\
\vdots\\
0\\
1
\end{bmatrix} u_x
\end{equation}
 However, just as with one robot, we can only control two states because $\mathcal{C}$ has rank two:
\begin{equation}
\mathcal{C}=\left\{ \begin{bmatrix} 
0\\
1\\
\vdots\\
0\\
1
\end{bmatrix}
,
  \begin{bmatrix} 
1\\
0\\
\vdots\\
1\\
0
\end{bmatrix}
,
\begin{bmatrix} 
0\\
0\\
\vdots\\
0\\
0
\end{bmatrix}, 
\ldots \right\}
\end{equation}  
\subsection{Controlling Mean Position}\label{sec:controlMeanPosition}
This means \emph{any} number of robots controlled by a global command have just two controllable states in each axis. We can not control the position of all the robots, but what states are controllable? To answer this question we create the following reduced order system that represents the average position and velocity of the $n$ robots:
\begin{align}
\begin{bmatrix}\nonumber
\dot{\bar{x}}_p \\
\dot{\bar{x}}_v
\end{bmatrix} &= \frac{1}{n} \begin{bmatrix}
0& 1& 0& 1& \ldots &0& 1 \\
0& 0& 0& 0& \ldots &0& 0
\end{bmatrix}
\begin{bmatrix}
x_1\\
x_2\\
\vdots\\
x_{2n-1}\\
x_{2n}
\end{bmatrix} \\
&+ \frac{1}{n}\begin{bmatrix}
0& 0& 0& 0& \ldots &0& 0 \\
0& 1& 0& 1& \ldots &0& 1
\end{bmatrix}\begin{bmatrix} 
0\\
1\\
\vdots\\
0\\
1
\end{bmatrix} u_x
\end{align}
Thus:
\begin{equation}
\begin{bmatrix}
\dot{\bar{x}}_p \\
\dot{\bar{x}}_v
\end{bmatrix} = \begin{bmatrix}
0& 1 \\
0& 0
\end{bmatrix}
\begin{bmatrix}
\bar{x}_p\\
\bar{x}_v
\end{bmatrix} + \begin{bmatrix} 
0\\
1
\end{bmatrix} u
\end{equation}

We again analyze $\mathcal{C}$:
\begin{equation}
\mathcal{C}=\left\{ \begin{bmatrix} 
0\\
1
\end{bmatrix}
,
 \begin{bmatrix} 
1\\
0
\end{bmatrix}
 \right\}
\end{equation}
This matrix has rank two, and thus the controllable states of the swarm are the average position and average velocity.
%\begin{equation}
%a = K\begin{bmatrix}
%\begin{bmatrix}
%\dot{\bar{x}}_p \\
%\dot{\bar{x}}_v
%\end{bmatrix}
%-
%\begin{bmatrix}
%x_{goal} \\
%\dot{x}_{goal}
%\end{bmatrix}
%\end{bmatrix}
%\end{equation}
\hfill $\blacksquare$ 

Due to symmetry, only the mean position and mean velocity are controllable. However, there are several techniques for breaking symmetry, for example by allowing independent noise sources \cite{beckerIJRR2014}, or by using obstacles \cite{Becker2013b}.

\subsection{Controlling the variance of many robots}\label{sec:VarianceControl}

The variance, $\sigma^2$, of the robot's position is computed:
\begin{align}
 \overline{x}(\mathbf{x}) = \frac{1}{n} \sum_{i=1}^n x_i, \qquad  %\nonumber \\ 
\sigma^2(\mathbf{x}) &= \frac{1}{n} \sum_{i=1}^n (x_i - \overline{x})^2.  %NOT 1/(N-1) because we are measuring the actual Variance, not the variance of samples drawn from a distribution
\end{align}

Controlling the variance requires being able to increase and decrease the variance.  We will list a sufficient condition for each. Both conditions are readily found at the micro and nanoscale. 
Real systems, especially at the micro scale, are affected by unmodelled dynamics. These unmodelled dynamics are dominated by Brownian noise. To model this~\eqref{eq:stdform} must be modified as follows:
\begin{align}
\dot{\mathbf{x}}(t)  &=  A \mathbf{x}(t) + B \mathbf{u}(t) + W \bm{\varepsilon}(t)\\
 \mathbf{y}(t) &= C  \mathbf{x}(t) + D  \mathbf{u}(t)\nonumber
\end{align}
where $W\bm{\varepsilon}(t)$ is a random perturbation produced by Brownian noise. Given a large free workspace, a \emph{Brownian noise} process increases the variance linearly with time.
\begin{equation*}\dot{\sigma}^2(\mathbf{x}(t), \mathbf{u}(t) = 0)  = W \bm{\varepsilon} \end{equation*}
%After some time, Gaussian distribution shapes the outline of the robots because of the Brownian noise feature:\\
%\begin{equation}
%P(x) = \frac{1}{\sigma \sqrt{2\pi } }e^{ -  \frac{\left( x - \mu  \right)^2 } {2\sigma ^2 }}
%\end{equation}
If the robots are in a bounded environment with sides of length $[\ell_x, \ell_y]$, the unforced variance asymptotically grows to the variance of a uniform distribution,
\begin{align}
[\sigma_x^2,\sigma_y^2] = \frac{1}{12}[ (\ell_x - 2 r)^2,(\ell_y - 2 r)^2].\label{eq:VarianceUniformDistribution}
\end{align}
 Therefore, to achieve a desired variance $\sigma^2_{goal}$, the swarm's mean position must be at least $r+\sqrt{3 \sigma^2_{goal}}$ from the nearest wall.

 A flat obstacle can be used to decrease variance. Pushing a group of dispersed robots against a flat obstacle will decrease their variance until the minimum-variance (maximum density) packing  is reached. For large $n$, Graham and Sloan showed that the minimum-variance packing  $\sigma^2_{optimal}(n,r)$ for $n$ circles with radius $r$ is $\approx 
  n r^2 \frac{\sqrt{3}}{2 \pi} = 0.28$~\cite{graham1990penny}. 
% $\approx 4 r^2 (\sqrt{3} n^2)/(4 \pi)=0.138 n^2 r^2$
If we only need to reduce variance in a single axis, the variance can be reduced to zero, given sufficient space.

We will prove the origin is globally asymptotically stabilizable by using a control-Lyapunov function \cite{Lyapunov1992}.  A suitable Lyapunov function is squared variance error:
\begin{align}
\label{eq:LyapunovVariance}
V(t,\bf{x})  &= \frac{1}{2} (\sigma^2(\mathbf{x}) - \sigma^2_{goal})^2\\
\dot{V}(t,\bf{x}) &= (\sigma^2(\mathbf{x})-\sigma^2_{goal})\dot{\sigma}^2(\mathbf{x})
\end{align}
We note here that $V(t,\mathbf{x})$ is positive definite and radially unbounded, and $V(t,\mathbf{x}) \equiv 0$ only at $\sigma^2(\mathbf{x}) = \sigma^2_{goal}$.
To make $\dot{V}(t,\mathbf{x})$ negative semi-definite, we choose
\begin{align}
u(t) &=   \begin{cases}
	 \mbox{move to wall} &\mbox{if } \sigma^2(\mathbf{x})>\sigma^2_{goal} \\ 
	 \mbox{move from wall} & \mbox{if } \sigma^2(\mathbf{x}) \le \sigma^2_{goal}.
\end{cases} 
\end{align}
 For such a $u(t)$,
 \begin{align}
\dot{\sigma}^2(\mathbf{x}) &=   \begin{cases}
	 \mbox{negative} &\mbox{if } \sigma^2(\mathbf{x})> \max(\sigma^2_{goal}, \sigma^2_{min}(n,r))  \\ 
	 W & \mbox{if } \sigma^2(\mathbf{x}) \le \sigma^2_{goal},
\end{cases} 
\end{align} and thus
$\dot{V}(t,{\bf x})$ is negative definite and the variance is globally asymptotically stabilizable. \hfill$\blacksquare$ 





\subsection{Controlling both the mean and the variance of many robots}

The mean and variance of the swarm cannot be controlled simultaneously, however if the dispersion due to Brownian motion is much less than the maximum controlled speed, we can adopt a hybrid, hysteresis-based controller to regulate the mean and variance shown in Alg.~\ref{alg:MeanVarianceControl}.  Such a controller normally controls the mean position according to \eqref{eq:PDcontrolPosition}, but switches to minimizing variance if the variance exceeds some $\sigma_{max}^2$.  The variance is lowered to less than $\sigma_{min}^2$, and the system returns to controlling the mean position. This is a standard technique for dealing with control objectives that evolve at different rates~\cite{Sadraddini2015,kloetzer2007temporal}, and the hysteresis avoids rapid switching between control modes. The process is illustrated in Fig.~\ref{fig:hysteresis}. 


\begin{algorithm}
\caption{Hybrid mean and variance control}\label{alg:MeanVarianceControl}
\begin{algorithmic}[1]
\Require Knowledge of swarm mean $[\bar{x},\bar{y}]$, the locations of the rectangular boundary $\{x_{min}, x_{max}, y_{min}, y_{max}\}$, and the target mean position $[x_{target},y_{target}]$.%TODO: use  \AND, \OR, \XOR, \NOT, \TO, \TRUE, \FALSE \gets
\State $flag_x \gets \FALSE$,  $flag_y \gets \FALSE$ 
\State $x_{goal} \gets  x_{target}$, $y_{goal} \gets y_{target}$
\Loop
\State  Compute $\sigma_x^2, \sigma_y^2$

\If {$\sigma_x^2 > \sigma_{max}^2$}
\State $x_{goal}  \gets x_{min}$
\State $flag_x  \gets \TRUE$
\Else {\bf~if} {$flag_x$ \textbf{and} $\sigma_x^2 < \sigma_{min}^2$}
\State $y_{goal}  \gets  y_{target}$
\State $flag_x  \gets false$
\EndIf

\If{$\sigma_y^2 > \sigma_{max}^2$}
\State $y_{goal}  \gets y_{min}$
\State $flag_y  \gets \TRUE$
\Else {\bf~if} {$flag_y$ \textbf{and} $\sigma_y^2 < \sigma_{min}^2$}
\State $y_{goal}  \gets  y_{target}$
\State $flag_y  \gets \FALSE$
\EndIf
\State Apply \eqref{eq:PDcontrolPosition} to move toward $[x_{goal}, y_{goal}]$
\EndLoop
\end{algorithmic}
\end{algorithm}
\begin{figure}
\centering
\begin{overpic}[scale=.3]{hysteresis.png}
\put(39,21){$\sigma^2 < \sigma^2_{min}$ }
\put(39,5){$\sigma^2 > \sigma^2_{max}$}\end{overpic}
\vspace{-1em}
\caption{\label{fig:hysteresis} Two states for controlling the mean and variance of a robot swarm.
%\vspace{-2em}
}
\end{figure}

A key challenge is to select proper values for $\sigma_{min}^2$ and $\sigma_{max}^2$.  The optimal packing variance is 
$ \sigma^2_{optimal}(n,r) = n r^2 \frac{\sqrt{3}}{2 \pi} $.  
The random packings generated by pushing our robots into corners are suboptimal, so we choose the conservative values shown in Fig.~\ref{fig:VarianceMinMaxBand}:
\begin{align}
 \sigma^2_{min} &= 2.5r+ \sigma^2_{optimal}(n,r) \nonumber\\
  \sigma^2_{max} &= 15r+ \sigma^2_{optimal}(n,r).
  \end{align}
%\sigma \approx 0.371258 n r,   \sigma_{min} = n*r*1/2, \sigma_{min} = n*r*3/2, 
%  this result was verified in http://www-math.mit.edu/~tchow/penny.pdf

\begin{figure}
\centering
\begin{overpic}[width = \columnwidth]{VarianceMinMaxBand2.pdf}\end{overpic}
\vspace{-1em}
\caption{\label{fig:VarianceMinMaxBand} The switching conditions for variance control are set as a function of $n$, and designed to be larger than the optimal packing density. The above plot uses robot radius $r=1/10$.
}\vspace{-1em}
\end{figure}



%what images should I show here?
%hysteresis control law, \cite{sadra2014}










