
\documentclass[conference]{IEEEtran}
\usepackage{times}

% numbers option provides compact numerical references in the text. 
\usepackage[numbers]{natbib}
\usepackage{multicol}
\usepackage[bookmarks=true]{hyperref}

\usepackage{bbm}
\usepackage{calc}
\usepackage{url}
\usepackage{hyperref}
\hypersetup{
  colorlinks =true,
  urlcolor = black,
  linkcolor = black
}
\usepackage{graphicx}
\usepackage[cmex10]{amsmath}
\usepackage{bm}
\usepackage{amssymb}
\usepackage{rotating}


%\usepackage{xfrac}
\usepackage{nicefrac}
\usepackage{cite}
\usepackage[caption=false,font=footnotesize]{subfig}
\usepackage[usenames, dvipsnames]{color}
\usepackage{colortbl}
%\usepackage{caption}

%\usepackage{wrapfig}
\usepackage{overpic}
%\usepackage{subfigure}
%\usepackage{textcomp}
\graphicspath{{./pictures/pdf/},{./pictures/ps/},{./pictures/png/},{./pictures/jpg/}}
\usepackage{breqn} %for breaking equations automatically
\usepackage[ruled]{algorithm}
\usepackage{algpseudocode}
%\usepackage{algorithmic}
\usepackage{multirow}
\usepackage{todonotes}

%\newcommand{\todo}[1]{\vspace{5 mm}\par \noindent \framebox{\begin{minipage}[c]{0.98 \columnwidth} \ttfamily\flushleft \textcolor{red}{#1}\end{minipage}}\vspace{5 mm}\par}
% uncomment this to hide all red todos
%\renewcommand{\todo}{}

%% ABBREVIATIONS
\newcommand{\qstart}{q_{\text{start}}}
\newcommand{\qgoal}{q_{\text{goal}}}
\newcommand{\pstart}{p_{\text{start}}}
\newcommand{\pgoal}{p_{\text{goal}}}
\newcommand{\xstart}{x_{\text{start}}}
\newcommand{\xgoal}{x_{\text{goal}}}
\newcommand{\ystart}{y_{\text{start}}}
\newcommand{\ygoal}{y_{\text{goal}}}
\newcommand{\gammastart}{\gamma_{\text{start}}}
\newcommand{\gammagoal}{\gamma_{\text{goal}}}
\providecommand{\proc}[1]{\textsc{#1}}


\newcommand{\ARLfull}{Aero\-space Ro\-bot\-ics La\-bora\-tory }
\newcommand{\ARL}{\textsc{arl}}
\newcommand{\JPL}{\textsc{jpl}}
\newcommand{\PRM}{\textsc{prm}}

\newcommand{\CM}{\textsc{cm}}
\newcommand{\SVM}{\textsc{svm}}
\newcommand{\NN}{\textsc{nn}}
\newcommand{\prm}{\textsc{prm}}
\newcommand{\lemur}{\textsc{lemur}}
\newcommand{\Lemur}{\textsc{Lemur}}
\newcommand{\LP}{\textsc{lp}} 
\newcommand{\SOCP}{\textsc{socp}}
\newcommand{\SDP}{\textsc{sdp}}
\newcommand{\NP}{\textsc{np}}
\newcommand{\SAT}{\textsc{sat}}
\newcommand{\LMI}{\textsc{lmi}}
\newcommand{\hrp}{\textsc{hrp\nobreakdash-2}}
\newcommand{\DOF}{\textsc{dof}}
\newcommand{\UIUC}{\textsc{uiuc}}
%% MACROS


\providecommand{\abs}[1]{\left\lvert#1\right\rvert}
\providecommand{\norm}[1]{\left\lVert#1\right\rVert}
\providecommand{\normn}[2]{\left\lVert#1\right\rVert_#2}
\providecommand{\dualnorm}[1]{\norm{#1}_\ast}
\providecommand{\dualnormn}[2]{\norm{#1}_{#2\ast}}
\providecommand{\set}[1]{\lbrace\,#1\,\rbrace}
\providecommand{\cset}[2]{\lbrace\,{#1}\nobreak\mid\nobreak{#2}\,\rbrace}
\providecommand{\lscal}{<}
\providecommand{\gscal}{>}
\providecommand{\lvect}{\prec}
\providecommand{\gvect}{\succ}
\providecommand{\leqscal}{\leq}
\providecommand{\geqscal}{\geq}
\providecommand{\leqvect}{\preceq}
\providecommand{\geqvect}{\succeq}
\providecommand{\onevect}{\mathbf{1}}
\providecommand{\zerovect}{\mathbf{0}}
\providecommand{\field}[1]{\mathbb{#1}}
\providecommand{\C}{\field{C}}
\providecommand{\R}{\field{R}}
\newcommand{\Cspace}{\mathcal{Q}}
\newcommand{\Uspace}{\mathcal{U}}
\providecommand{\Fspace}{\Cspace_\text{free}}
\providecommand{\Hcal}{$\mathcal{H}$}
\providecommand{\Vcal}{$\mathcal{V}$}
\DeclareMathOperator{\conv}{conv}
\DeclareMathOperator{\cone}{cone}
\DeclareMathOperator{\homog}{homog}
\DeclareMathOperator{\domain}{dom}
\DeclareMathOperator{\range}{range}
\DeclareMathOperator{\sign}{sgn}
\providecommand{\polar}{\triangle}
\providecommand{\ainner}{\underline{a}}
\providecommand{\aouter}{\overline{a}}
\providecommand{\binner}{\underline{b}}
\providecommand{\bouter}{\overline{b}}
\newcommand{\D}{\nobreakdash-\textsc{d}}
%\newcommand{\Fspace}{\mathcal{F}}
\providecommand{\Fspace}{\Cspace_\text{free}}
\providecommand{\free}{\text{\{}\mathsf{free}\text{\}}}
\providecommand{\iff}{\Leftrightarrow}
\providecommand{\subinner}[1]{#1_{\text{inner}}}
\providecommand{\subouter}[1]{#1_{\text{outer}}}
\providecommand{\Ppoly}{\mathcal{X}}
\providecommand{\Pproj}{\mathcal{Y}}
\providecommand{\Pinner}{\subinner{\Pproj}}
\providecommand{\Pouter}{\subouter{\Pproj}}
\DeclareMathOperator{\argmax}{arg\,max}
\providecommand{\Aineq}{B}
\providecommand{\Aeq}{A}
\providecommand{\bineq}{u}
\providecommand{\beq}{t}
\DeclareMathOperator{\area}{area}
\newcommand{\contact}[1]{\Cspace_{#1}}
\newcommand{\feasible}[1]{\Fspace_{#1}}
\newcommand{\dd}{\; \mathrm{d}}
\newcommand{\figwid}{0.22\columnwidth}
\newcommand{\TRUE}{\textbf{true}}
\newcommand{\FALSE}{\textbf{false}}
\DeclareMathOperator{\atan2}{atan2}


\newtheorem{theorem}{Theorem}
\newtheorem{definition}[theorem]{Definition}
\newtheorem{lemma}[theorem]{Lemma}


\pdfinfo{
   /Author (withheld for double-blind review)
   /Title  (Shaping a Swarm Using Wall Friction and a Shared Control Input)
   /CreationDate (D:20160129120000)
   /Subject (Simple Robots)
   /Keywords (Robots;Uniform Control Inputs)
}

\begin{document}

% paper title
\title{\large{ \emph{Supplement to} 
Shaping a Swarm With a Shared Control Input\\ Using Boundary Walls and Wall Friction}}

% You will get a Paper-ID when submitting a pdf file to the conference system
\author{Paper-ID [add your ID here]}


\maketitle

\begin{abstract}
Includes algorithms and equations too lengthy for main paper, but potentially useful for the community.
\end{abstract}

\IEEEpeerreviewmaketitle

\section{ Calculations for modeling swarm as fluid in a simple planar workspace}


\section{ Algorithm for generating desired $y$ spacing between two robots using wall friction}
\begin{algorithm}
\caption{GenerateDesired$y$-spacing($s_1,s_2,e_1,e_2,L$)}\label{alg:YControl}
\begin{algorithmic}[1]
\Require Knowledge of starting $(s_1,s_2)$ and ending $(e_1,e_2)$ positions of  two robots. 
$(0,0)$ is bottom corner, $s_1$ is rightmost robot, 
 $L$ is length of the walls. Current position of the robots are $(r_1,r_2)$.
\Ensure   $ r_{1x} - r_{2x}  \equiv s_{1x} - s_{2x} $   %$\Delta y(t) \equiv \Delta y(0)$ 
\State $ \Delta s_y  \gets s_{1y} - s_{2y} $
\State $ \Delta e_y \gets e_{1y} - e_{2y} $
\State $ r_1 \gets s_1$, $ r_2 \gets s_2$
\If {$\Delta e_y < 0 $ }
\State $ m \gets ( L-\max( r_{1y},r_{2y}) ,0)   $ \Comment{Move to top wall}
\Else 
\State  $ m \gets ( -\min( r_{1y},r_{2y}),0 )    $ \Comment{Move to bottom wall}
\EndIf
\State $m  \gets  m + (0, -\min( r_{1x},r_{2x} ))$ \Comment{Move to left}
\State $ r_1 \gets r_1+m$, $ r_2 \gets r_2+m$ \Comment{Apply move}
\If {$\Delta e_y - (r_{1y} - r_{2y} ) > 0 $}
\State $ m \gets (\min(|\Delta e_y - \Delta s_y |, L- r_{1y}), 0)$  \Comment{Move top}
\Else
\State $ m \gets (-\min(|\Delta e_y - \Delta s_y |, r_{1y}), 0)$\Comment{Move bottom}
\EndIf 
\State $m  \gets  m + (0, \epsilon)$ \Comment{Move right}
\State $ r_1 \gets r_1+m$, $ r_2 \gets r_2+m$ \Comment{Apply move}
\State $\Delta r_y = r_{1y} - r_{2y}$
\If {$\Delta r_y \equiv \Delta e_y$} 
\State   $ m \gets (e_{1x}-r_{1x}, e_{1y}-r_{1y})$
\State $ r_1 \gets r_1+m$, $ r_2 \gets r_2+m$ \Comment{Apply move}
\State  \Return $(r_1,r_2)$
\Else   
\State \Return GenerateDesired$y$-spacing($r_1,r_2,e_1,e_2,L$)
\EndIf
\end{algorithmic}
\end{algorithm}

%%%%%%%%%%%%%%%
%% Use plainnat to work nicely with natbib. 
\bibliographystyle{plainnat}
\bibliography{IEEEabrv,ShapingSwarmFrictionSharedInput,../../../RoboticSwarmControlLab/bib/aaronrefs}
\end{document}



