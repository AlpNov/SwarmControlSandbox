\section{Introduction}\label{sec:Intro}
Micro- and nano-robots can be manufactured in large numbers, see recent examples in~\citet{Donald2013}, \citet{Ghosh2009}, \citet{kim2015imparting}, or \citet{martel2014computer}.
Our vision is for large swarms of robots remotely guided 1) through the human body, to cure disease, heal tissue, and prevent infection  and 2) ex vivo to assemble structures in parallel. %Both tasks are high   as described in \citet{munoz2014review} \citet{sitti2015biomedical} . 
 For each application, large numbers of micro robots are required  to deliver sufficient payloads, but the small size of these robots makes it difficult to perform onboard computation.  Instead, these robots are often controlled by a global, broadcast signal. 
 The biggest barrier to this vision is a lack of control techniques that can reliably exploit large populations despite high under-actuation.  
 

\begin{figure}
\centering
\begin{overpic}[width=0.9\columnwidth]{Leaf2.jpg}\end{overpic}
\caption{\label{fig:vascularNetwork}
Vascular networks are common in biology such as the circulatory system and cerebrospinal spaces, as well as in porous media including sponges and pumice stone.  Navigating a swarm using global inputs, where each member receives the same control inputs, is challenging
due to the many obstacles. 
This paper focuses on using boundary walls and wall friction to break the symmetry caused by the global input and control the shape of a swarm.} %TODO:  1.) save this as a pdf 2.) use brighter colors.  These are hard to see.
\end{figure}

%One surprising result was that humans that only knew the swarm's mean and covariance completed the task faster that humans who knew the position of every robot~\cite{Becker2013b}. Our previous work focused on a block-pushing task, where a swarm of robots pushed a larger block through a 2D maze. 
Even without obstacles or boundaries, the mean position of a swarm is controllable.  By adding rectangular boundary walls, some higher-order moments such as the swarm's position variance orthogonal to the boundary walls ($\sigma_x$ and $\sigma_y$ for a workspace with axis-aligned walls) are also controllable. 
%The usefulness of these techniques was recently demonstrated by several automatic controllers. One controller steered a swarm of robots to push a larger block through a 2D maze~\citet{ShahrokhiIROS2015}. 
A limitation is that global control can only compress a swarm orthogonal to obstacles.  One implication is that a swarm in an axis-aligned rectangular workspace can not generate a non-zero covariance. This limitation is detrimental to desired applications because the ability to orient the swarm is often useful for navigating narrow passages.
%This means the swarm could not navigate workspaces with narrow corridors with other orientations, such as those shown in Fig.\ \ref{fig:covFriction}.
%Challenges like these require a controller that regulates the swarm's position covariance, $\sigma_{xy}$. 

After a review of recent related work (Section \ref{sec:RelatedWork}), this paper provides analytical position control results in two canonical workspaces with frictionless walls (Section \ref{subsec:FluidInTank}).  These results are limited in the set of shapes that can be controlled.  To extend the range of possible shapes, we introduce wall friction to the system model (Section \ref{subsec:WallFriction}).  We prove that two orthogonal boundaries with high friction are sufficient to arbitrarily position two robots (Section \ref{sec:PostionControl2Robots}), extend this proof to prove that a rectangular workspace with high friction boundaries is sufficient to position a swarm of $n$ robots arbitrarily within a subset of the workspace (Section \label{sec:PostionControlnRobots}), implement both position control algorithms in simulation and on a hardware setup with up to 97 robots (Sections \ref{sec:simulation} and \ref{sec:expResults}), and end with directions for future research (Section )\label{sec:conclusion}.
%For controlling $\sigma_{xy}$, we prove that the swarm position covariance $\sigma_{xy}$ is controllable given boundaries with non-zero friction. 
%We then prove that two orthogonal boundaries with high friction are sufficient to arbitrarily position a swarm of $n$ robots. 
%We conclude by designing controllers that efficiently regulate $\sigma_{xy}$.
%This paper
%(1) proves that the swarm position covariance $\sigma_{xy}$ is controllable given boundaries with non-zero friction,  %where do we do this?
%(2) proves that two orthogonal boundaries with high friction are sufficient to arbitrarily position two robots, 
%(3) proves that two orthogonal boundaries with high friction are sufficient to arbitrarily position a swarm of $n$ robots, 
%(4) shows full-state position control with 2 or more robots using  extensive simulations, and
%(5) demonstrate covariance control on our hardware platform with a large number of hardware robots.
%TODO JOURNAL: design controllers that efficiently regulate $\sigma_{xy}$.
%TODO JOURNAL: We will design Lyapunov-inspired controllers for $\sigma_{xy}$ to prove controllability. 
%TODO JOURNAL:  and rank controllability as a function of friction.
% TODO: JOURNAL: and vary wall friction by laser-cutting boundary walls with a variety of profiles. 
%\begin{figure}[t]
%\centering
%\begin{overpic}[width = \columnwidth]{Covariance.jpg}\end{overpic}
%\vspace{-1em}
%\caption{\label{fig:covFriction} Maintaining group cohesion while steering a swarm through an arbitrary maze requires covariance control.
%}\vspace{-1em}
%\end{figure}
% Our paper is organized as follows.  After a discussion of related work in Section \ref{sec:RelatedWork}, we describe our experimental methods for an online human-user experiment in Section \ref{sec:expMethods}.  We report the results of our experiments in Section \ref{sec:expResults}, discuss the lessons learned in Section \ref{sec:discussion}, and end with concluding remarks in Section \ref{sec:conclusion}.