%%%%%%%%%%%%%%%%%%%%%%%%%%%%%%%%%%%%%%%%%%%%%%%%%%%%%%%%%%%
\section{Theory}
\label{sec:theory}
%%%%%%%%%%%%%%%%%%%%%%%%%%%%%%%%%%%%%%%%%%%%%%%%%%%%%%%%%%%


% Theory:
%% Models (of each robot, of the global input)
%% Impossibility result
%% Controlling Mean proof
%% Controlling Variance proof (CLF)
%% A Hybrid controller using hysteresis
\subsection{Models}
We consider holonomic robots that move in the 2D plane.
We want to control position and velocity of the robots and deciding about force for reaching our desired position and velocity. So our input is going to be acceleration (\emph{F = ma}). If we show acceleration by \emph{a}, velocity by \emph{v} and position in x coordinate by \emph{$P_x$} and in y coordinate by \emph{$P_y$}, we have the following equations
\begin{equation}
\begin{bmatrix}
\dot{P}_x = v_x \\
\dot{v}_x = a_x
\end{bmatrix}
\end{equation}
\begin{equation}
\begin{bmatrix}
\dot{P}_y = v_y \\
\dot{v}_y = a_y
\end{bmatrix}
\end{equation}


The state-space representation of our OpenLoop controller is: 
\begin{align}
\dot{x}(t)  &=  A x(t) + B u(t) \\
y &= C x(t) + D u(t)\nonumber
\end{align}


where \emph{x(t)} represents our states, \emph{u(t)} is our input and \emph{e(t)} represents noise in the system. We also have \emph{y} as our output.\\
First we assume that we don't have noise in the system and we also have just one robot. As mentioned, \emph{a} is our input and \emph{x, y,} and their velocities are our states that we want to control.\\
We define our states as following:
\begin{align}
x_1 &= P_x \\ \nonumber
\dot{x}_1 &= x_2 = \dot{P}_x = v_x\\\nonumber
x_3 &= P_y\\ \nonumber
 \dot{x}_3 &= x_4 = \dot{P}_y = v_y \nonumber
\end{align}


So our state space representation is:
\begin{equation}
\begin{bmatrix}
\dot{x}_1\\ 
\dot{x}_2\\
\dot{x}_3\\
\dot{x}_4
\end{bmatrix} = \begin{bmatrix}
0 & 1 & 0 & 0 \\
0 & 0 & 0 & 0\\
0 & 0 & 0 & 1\\
0 & 0 & 0 & 0
\end{bmatrix}  \begin{bmatrix}
x_1\\
x_2\\
x_3\\
x_4
\end{bmatrix} + \begin{bmatrix}
0 & 0 \\
1 & 0 \\
0 & 0 \\
0 & 1
\end{bmatrix} u
\end{equation}

We want to find number of states that we can control. We need to know rank of the controllability matrix \emph{\{B, AB,$A^2$B, ... , $A^{n-1}$B\}}.
\begin{equation}
C = \{ B, AB, A^2B, ... , A^{n-1}B \}
\end{equation}

\begin{equation}
C=\left\{
\begin{bmatrix} 
0 & 0\\
1 & 0 \\
0 & 0 \\
0 & 1
\end{bmatrix}
,
\begin{bmatrix} 
1 & 0\\
0 & 0\\
0 & 1\\
0 & 0
\end{bmatrix}
\begin{bmatrix} 
0 & 0\\
0 & 0\\
0 & 0\\
0 & 0
\end{bmatrix}
,
\begin{bmatrix} 
0 & 0\\
0 & 0\\
0 & 0\\
0 & 0
\end{bmatrix}
 \right\}
\end{equation}
So we showed that we have two controllable states.

\subsection{Independent control is not possible}
We know that we can control velocity of our robot, we want to see what happens if we had more than one robot? As we saw here, $v_x$ is completely independent of $v_y$, so if we wanted to have n robots, we had n states in x axis, and n states in y axis, which were completely independent from each other. So assume we have n robots and want to control them in \emph{x} axis:\\
\begin{align}
\dot{P}_{x1} &= v_{x1}\\\nonumber
\dot{v}_{x1} &= a_{x1}\\\nonumber
\dot{P}_{x2} &= v_{x2}\\\nonumber
\dot{v}_{x2} &= a_{x2}\\\nonumber
&\vdots\\\nonumber
\dot{P}_{xn} &= v_{xn}\\\nonumber
\dot{v}_{xn} &= a_{xn}\nonumber
\end{align}


So our state-space representation will be:
\begin{equation}
\begin{bmatrix}
\dot{x}_1\\ 
\dot{x}_2\\
.\\
.\\
.\\
\dot{x}_{2n-1}\\
\dot{x}_{2n}

\end{bmatrix} = \begin{bmatrix}
0 & 1 & . & . & . & 0 & 0 \\
0 & 0 & . & . & . & 0 & 0 \\
0 & 0 & 0 & 1 & . & 0 & 0 \\
0 & 0 & 0 & 0 & . & 0 & 0 \\
. & . & . & . & . & . & . \\
0 & 0 & . & . & . & 0 & 1 \\
0 & 0 & . & . & . & 0 & 0 
\end{bmatrix}  \begin{bmatrix}
x_1\\
x_2\\
.\\
.\\
.\\
x_{2n-1}\\
x_{2n}
\end{bmatrix} + \begin{bmatrix}
0\\
1\\
.\\
.\\
.\\
0\\
1
\end{bmatrix} a_x
\end{equation}

If we had n robots, we had exactly the same symmetry of what we had for 1 robot. We can again control two states because we have rank two in C:
\begin{equation}
C=\left\{ \begin{bmatrix} 
0\\
1\\
.\\
.\\
.\\
0\\
1
\end{bmatrix}
,
  \begin{bmatrix} 
1\\
0\\
.\\
.\\
.\\
1\\
0
\end{bmatrix}
,
\begin{bmatrix} 
0\\
0\\
.\\
.\\
.\\
0\\
0
\end{bmatrix}, ... \right\}
\end{equation}  
\subsection{Controlling Mean Position}
So for any number of robots if we give a global command to them, we have just two controllable states in each axis. So it is obvious that we can not control position of all the robots, but what states are controllable? To answer this question we create a reduced order system that calculates average position and average velocity of the robots:\\

\begin{equation}
\begin{bmatrix}
\dot{\bar{x}}_p \\
\dot{\bar{x}}_v
\end{bmatrix} = \frac{1}{n} \begin{bmatrix}
0& 1& 0& 1& ... &0& 1 \\
0& 0& 0& 0& ... &0& 0
\end{bmatrix}
\begin{bmatrix}
x_1\\
x_2\\
.\\
.\\
.\\
x_{2n-1}\\
x_{2n}
\end{bmatrix} + \frac{1}{n}\begin{bmatrix}
0& 0& 0& 0& ... &0& 0 \\
0& 1& 0& 1& ... &0& 1
\end{bmatrix}\begin{bmatrix} 
0\\
1\\
.\\
.\\
.\\
0\\
1
\end{bmatrix} u_x
\end{equation}
Thus:
\begin{equation}
\begin{bmatrix}
\dot{\bar{x}}_p \\
\dot{\bar{x}}_v
\end{bmatrix} = \begin{bmatrix}
0& 1 \\
0& 0
\end{bmatrix}
\begin{bmatrix}
\bar{x}_p\\
\bar{x}_v
\end{bmatrix} + \begin{bmatrix} 
0\\
1
\end{bmatrix} u
\end{equation}

We analyze \emph{C} for \emph{y}:
\begin{equation}
C=\left\{ \begin{bmatrix} 
0\\
1
\end{bmatrix}
,
 \begin{bmatrix} 
1\\
0
\end{bmatrix}
 \right\}
\end{equation}

This matrix again  has rank two, and thus all the states are controllable. These controllable states are the average position and average velocity:
\begin{equation}
a = K\begin{bmatrix}
\begin{bmatrix}
\dot{\bar{x}}_p \\
\dot{\bar{x}}_v
\end{bmatrix}
-
\begin{bmatrix}
x_{goal} \\
\dot{x}_{goal}
\end{bmatrix}
\end{bmatrix}
\end{equation}
$\blacksquare$ \\

\subsection{Controlling the variance of many robots}

As shown in section ?? above, only the mean position of a group of robots is controllable. However, there are several techniques for breaking symmetry, for example  by allowing independent noise sources \ref{beckerIJRR2014}, or by using obstacles \ref{beckerMassiveManipulation}.

Control the variance requires being able to increase and decrease the variance.  Given a large free workspace, Brownian noise is sufficient to increase the variance.  A flat obstacle can be used to decrease variance.  

Control Lyapunov function on $\sigma(t), \sigma_{goal}$, provide a control law.


Real systems, especially at the micro scale, are affected by unmodelled dynamics much of which can be designed by Brownian noise. To model this equation (3) must be modified as follows:
\begin{align}
\dot{x}(t)  &=  A x(t) + B u(t) + We(t)\\
y &= C x(t) + D u(t)\nonumber
\end{align}
where $e(t)$ is the error in the system.

After some time, gaussian distribution shapes the outline of the robots because of the Brownian noise feature:\\
\begin{equation}
P(x) = \frac{1}{{\sigma \sqrt {2\pi } }}e^{{{ - \left( {x - \mu } \right)^2 } \mathord{\left/ {\vphantom {{ - \left( {x - \mu } \right)^2 } {2\sigma ^2 }}} \right. \kern-\nulldelimiterspace} {2\sigma ^2 }}}
\end{equation}

where $\sigma$ is standard deviation:
\begin{equation}
\sigma = \sqrt{\frac{1}{N-1} \sum_{i=1}^N (x_i - \overline{x})^2}
\end{equation}

\subsection{Controlling both the mean and the variance of many robots}

When we want to control mean and variance in the same time, we should use a hysteresis control law for the variance along controlling mean. It means that we control mean position while variance is not bigger than our maximum desired variance. If variance was bigger than the maximum desired variance, we then stop controlling mean position and reduce the variance. when we reach some minimum desired variance, we go back to our mean position. 
hysteresis control law, \cite{sadra2014}

\begin{equation}
\label{eq:hysteresisControlLaw}
\end{equation}







