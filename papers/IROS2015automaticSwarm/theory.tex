%%%%%%%%%%%%%%%%%%%%%%%%%%%%%%%%%%%%%%%%%%%%%%%%%%%%%%%%%%%
\section{Theory}
\label{sec:theory}
%%%%%%%%%%%%%%%%%%%%%%%%%%%%%%%%%%%%%%%%%%%%%%%%%%%%%%%%%%%


% Theory:
%% Models (of each robot, of the global input)
%% Impossibility result
%% Controlling Mean proof
%% Controlling Variance proof (CLF)
%% A Hybrid controller using hysteresis
\subsection{Models}
Consider holonomic robots that move in the 2D plane. We want to control position and velocity of the robots. 
First, assume a noiseless system and with just one robot.
 Our inputs are global forces $[u_x,u_y]$. \begin{equation}
\begin{bmatrix}
\dot{p}_x &=& v_x \\
\dot{v}_x &=& \frac{1}{m} u_x
\end{bmatrix},
\begin{bmatrix}
\dot{p}_y &=& v_y \\
\dot{v}_y &=&\frac{1}{m} u_y
\end{bmatrix}
\end{equation}

The state-space representation in standard form is: 
\begin{align}\label{eq:stdform}
\dot{x}(t)  &=  A x(t) + B u(t) \\
y(t) &= C x(t) + D u(t)\nonumber \end{align}


%where $x(t)$ represents our states, \emph{u(t)} is our input. %and \emph{e(t)} represents noise in the system.  We also have $y(t)$ as our output.\\
We define our state vector as:
\begin{align}
\left[ x_1,x_2,x_3,x_4\right]^\intercal = \left[ p_x,v_x,p_y,v_y\right]^\intercal, \nonumber
\end{align}

and our state space representation as:
\begin{equation}
\begin{bmatrix}
\dot{x}_1\\ 
\dot{x}_2\\
\dot{x}_3\\
\dot{x}_4
\end{bmatrix} = \begin{bmatrix}
0 & 1 & 0 & 0 \\
0 & 0 & 0 & 0\\
0 & 0 & 0 & 1\\
0 & 0 & 0 & 0
\end{bmatrix}  \begin{bmatrix}
x_1\\
x_2\\
x_3\\
x_4
\end{bmatrix} + \begin{bmatrix}
0 & 0 \\
\frac{1}{m} & 0 \\
0 & 0 \\
0 & \frac{1}{m}
\end{bmatrix} u
\end{equation}

We want to find number of states that we can control, which is given by the rank of the \emph{controllability matrix}
\begin{equation}
\mathcal{C} = \{ B, AB, A^2B, ... , A^{n-1}B \}.
\end{equation}

\begin{equation}
\textrm{Here }
\mathcal{C}=\left\{
\begin{bmatrix} 
0 & 0\\
\frac{1}{m} & 0 \\
0 & 0 \\
0 & \frac{1}{m}
\end{bmatrix}
,
\begin{bmatrix} 
\frac{1}{m}& 0\\
0 & 0\\
0 & \frac{1}{m}\\
0 & 0
\end{bmatrix},
\begin{bmatrix} 
0 & 0\\
0 & 0\\
0 & 0\\
0 & 0
\end{bmatrix}
, \ldots
 \right\}
\end{equation}
And thus all four states are controllable

\subsection{Independent control with multiple robots is impossible}
A single robot is fully controllable, but what happens with $n$ robots? For holonomic robots, movement in the $x$ and $y$ coordinates are independent, so for notational convenience without loss of generality we will focus only on movement in the $x$ axis. Given $n$ robots to be  controlled in the $x$ axis, there are $2n$ states, $n$ positions and $n$ velocities.
\begin{equation}\left[ x_1,x_2,\ldots, x_{2n-1},x_{2n}\right]^\intercal = \left[ p_{x,1},v_{x,1},\ldots,p_{x,n},v_{x,n}\right]^\intercal \nonumber\end{equation}
%\begin{align}
%\dot{p}_{x1} &= v_{x1}\\\nonumber
%\dot{v}_{x1} &= a_{x1}\\\nonumber
%\dot{p}_{x2} &= v_{x2}\\\nonumber
%\dot{v}_{x2} &= a_{x2}\\\nonumber
%&\vdots\\\nonumber
%\dot{p}_{xn} &= v_{xn}\\\nonumber
%\dot{v}_{xn} &= a_{xn}\nonumber
%\end{align}
Our state-space representation is:
\begin{equation}
\begin{bmatrix}
\dot{x}_1\\ 
\dot{x}_2\\
\vdots\\
\dot{x}_{2n-1}\\
\dot{x}_{2n}

\end{bmatrix} = \begin{bmatrix}
0 & 1 & \ldots & 0 & 0 \\
0 & 0 & \ldots& 0 & 0 \\
\vdots &  \vdots & \ddots & \vdots & \vdots \\
0 & 0  & \ldots & 0 & 1 \\
0 & 0 & \ldots& 0 & 0 
\end{bmatrix}  \begin{bmatrix}
x_1\\
x_2\\
\vdots \\
x_{2n-1}\\
x_{2n}
\end{bmatrix} + \begin{bmatrix}
0\\
1\\
\vdots\\
0\\
1
\end{bmatrix} u_x
\end{equation}
 However, just as with one robot, we can only control two states because $\mathcal{C}$ has rank two:
\begin{equation}
\mathcal{C}=\left\{ \begin{bmatrix} 
0\\
1\\
\vdots\\
0\\
1
\end{bmatrix}
,
  \begin{bmatrix} 
1\\
0\\
\vdots\\
1\\
0
\end{bmatrix}
,
\begin{bmatrix} 
0\\
0\\
\vdots\\
0\\
0
\end{bmatrix}, ... \right\}
\end{equation}  
\subsection{Controlling Mean Position}\label{sec:controlMeanPosition}
This means any number of robots controlled by a global command, have just two controllable states in each axis. We can not control position of all the robots, but what states are controllable? To answer this question we create the following reduced order system that represents the average position and velocity of the $n$ robots:\\

\begin{align}
\begin{bmatrix}\nonumber
\dot{\bar{x}}_p \\
\dot{\bar{x}}_v
\end{bmatrix} &= \frac{1}{n} \begin{bmatrix}
0& 1& 0& 1& ... &0& 1 \\
0& 0& 0& 0& ... &0& 0
\end{bmatrix}
\begin{bmatrix}
x_1\\
x_2\\
\vdots\\
x_{2n-1}\\
x_{2n}
\end{bmatrix} \\
&+ \frac{1}{n}\begin{bmatrix}
0& 0& 0& 0& ... &0& 0 \\
0& 1& 0& 1& ... &0& 1
\end{bmatrix}\begin{bmatrix} 
0\\
1\\
\vdots\\
0\\
1
\end{bmatrix} u_x
\end{align}
Thus:
\begin{equation}
\begin{bmatrix}
\dot{\bar{x}}_p \\
\dot{\bar{x}}_v
\end{bmatrix} = \begin{bmatrix}
0& 1 \\
0& 0
\end{bmatrix}
\begin{bmatrix}
\bar{x}_p\\
\bar{x}_v
\end{bmatrix} + \begin{bmatrix} 
0\\
1
\end{bmatrix} u
\end{equation}

We again analyze $\mathcal{C}$:
\begin{equation}
\mathcal{C}=\left\{ \begin{bmatrix} 
0\\
1
\end{bmatrix}
,
 \begin{bmatrix} 
1\\
0
\end{bmatrix}
 \right\}
\end{equation}
This matrix has rank two, and thus the controllable states of the swarm are the average position and average velocity.
%\begin{equation}
%a = K\begin{bmatrix}
%\begin{bmatrix}
%\dot{\bar{x}}_p \\
%\dot{\bar{x}}_v
%\end{bmatrix}
%-
%\begin{bmatrix}
%x_{goal} \\
%\dot{x}_{goal}
%\end{bmatrix}
%\end{bmatrix}
%\end{equation}
$\blacksquare$ 

\subsection{Controlling the variance of many robots}\label{sec:VarianceControl}

As shown in Section \ref{sec:controlMeanPosition}, only the mean position mean velocity is controllable. However, there are several techniques for breaking symmetry, for example by allowing independent noise sources \cite{beckerIJRR2014}, or by using obstacles \cite{Becker2013b}.

Controlling the variance requires being able to increase and decrease the variance.  We will list sufficient conditions for each of these.  Given a large free workspace, \emph{Brownian noise} is sufficient to increase the variance.   A flat obstacle can be used to decrease variance. Both conditions are readily found at the micro and nanoscale. The variance, $\sigma^2$, of the robot's position is computed:
\begin{equation}
\sigma = \frac{1}{N-1} \sum_{i=1}^N (x_i - \overline{x})^2
\end{equation}

Consider the following Control Lyapunov Function:
\begin{equation}
V = \frac{1}{2} (\sigma_{goal} - \sigma(x))^2
\end{equation}
If we calculate the derivative of Lyapunov function we have:

\todo{Aaron: finish this section}


\begin{equation}
\dot{V} = (\sigma_{goal} - \sigma(x)) \dot{\sigma}
\end{equation}



%Control Lyapunov function on $\sigma(t), \sigma_{goal}$, provide a control law.


Real systems, especially at the micro scale, are affected by unmodelled dynamics much of which can be designed by Brownian noise. To model this~\eqref{eq:stdform} must be modified as follows:
\begin{align}
\dot{x}(t)  &=  A x(t) + B u(t) + W \varepsilon(t)\\
y(t) &= C x(t) + D u(t)\nonumber
\end{align}
where $\varepsilon(t)$ is the error in the system.

After some time, Gaussian distribution shapes the outline of the robots because of the Brownian noise feature:\\
\begin{equation}
P(x) = \frac{1}{\sigma \sqrt{2\pi } }e^{ -  \frac{\left( x - \mu  \right)^2 } {2\sigma ^2 }}
\end{equation}

\subsection{Controlling both the mean and the variance of many robots}

The mean and variance of the swarm cannot be controlled simultaneously, however if the dispersion due to Brownian motion is much lass than the maximum controlled speed, we can adopt a hysteresis-based controller.  Such a controller normally controls the mean position according to \eqref{eq:PDcontrolPosition}, but switches to controlling variance if the variance exceeds some $\sigma_{max}^2$.  The variance is lowered to less than $\sigma_{min}^2$, and the system returns to controlling the mean position.
\begin{equation}
switch = \left\{ \begin{matrix} 
1 &\rightarrow & 2 & \sigma^2 & > & \sigma^2_{max} \\
2 & \rightarrow & 1 & \sigma^2 & < & \sigma^2_{min} 
\end{matrix}  \right.
\end{equation}
%what images should I show here?
%hysteresis control law, \cite{sadra2014}
This is a standard technique for dealing with control objectives that evolve at different rates~\cite{Sadraddini2015,kloetzer2007temporal}, and the hysteresis avoids rapid switching between control modes.

\begin{figure}
\centering
\begin{overpic}[scale=.3]{hysteresis.png}
\put(35,25){$\sigma^2 > \sigma^2_{max}$ }
\put(35,7){$\sigma^2 < \sigma^2_{min}$}\end{overpic}
\vspace{-1em}
\caption{\label{fig:hysteresis} Two states for controlling variances.
%\vspace{-2em}
}
\end{figure}







